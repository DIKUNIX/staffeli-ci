\documentclass[a4paper]{article}

\usepackage{notes}

\header{%
  assignment={Towards a Continuous Integration of \\ Canvas LMS REST API Scripts},%
  authors={DIKUNIX \texttt{<dikunix@dikumail.dk>}},%
  shortAuthors={\ },%
  date={May 3, 2017}
}

\begin{document}

\maketitle

\pagestyle{empty}

\bigskip

Canvas LMS has a REST
API\footnote{\url{https://canvas.instructure.com/doc/api/index.html}}.

\bigskip

The REST API allows for reliable automation (scripting) of tasks that would
otherwise be a laborious human chore when using the web-browser-based user
interface.  For instance, sorting students into sections and groups based on
some given criteria, batch download of student submissions, or batch upload of
feedback and grades. In general, the REST API enables automated course
management and grading.

Using the REST API is not straight-forward for all teaching staff, so to
leverage the benefits that it can provide across many courses, scripts written,
battle-tested, and hot-fixed on one course can be shared with another. Although
battle-testing and hot-fixing is great, it is far from a viable software
development methodology. We would like to see these scripts become subject to
continuous integration with automated testing.

Testing such scripts is currently \emph{possible}: Manually create a hidden
course. Add some willing \emph{real} system users as dummy students, teaching
assistants, and teachers. Finally, in the interest of not overwhelming the
current system setup, manually conduct small-scale tests while coordinating
with said \emph{real} users. This is a laborious human chore, and it does not
scale.

Hence, we would like to make the following requests:

\begin{enumerate}

\item Would it be possible to set up a sandboxed environment, which is morally
equivalent to the current, publicly facing Canvas LMS setup at the University
of Copenhagen, where we can conduct such automated tests of our automation
scripts?

\item If not, the main shortcoming of the manually-created-course approach is
the need to use real users. Would it be possible to get a dedicated pool of
dummy users for such purposes?

Also, to what extent can we abuse such manually created courses? Is there a
limit on how many we can have, how many times we can create sections, groups,
assignments, etc. within such courses?

\end{enumerate}

\end{document}
